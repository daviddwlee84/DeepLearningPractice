\documentclass[12pt]{article}
\usepackage[utf8]{inputenc}

\usepackage{mathptmx}

% To show Chinese characters
\usepackage{xeCJK}
\setCJKmainfont{SimSun}

\usepackage{geometry}
\geometry{a4paper, scale=0.8}

% For prettier tables
% replace \hline with \toprule, \midrule and \bottomrule
\usepackage{booktabs}
% for multi-row table
\usepackage{multirow}

% For list spacing
\usepackage{enumitem}
\setlist{noitemsep}

% For paragraph multi-column
\usepackage{multicol}
\setlength{\columnsep}{1cm}

% For graphic/figure
\usepackage{graphicx}

% For url in footnote, hyperlinks and bookmarks
\usepackage{hyperref}
\hypersetup{
    colorlinks=true,
    linkcolor=blue,
    filecolor=magenta,
    urlcolor=cyan,
    pdftitle={Sequence Labeling: Chinese Word Segmentation and Named Entity Recognition},
    bookmarks=true,
    pdfpagemode=FullScreen,
}
\newcommand\fnurl[2]{
  \href{#2}{#1}\footnote{\url{#2}}
}

\title{\textbf{Sequence Labeling: Chinese Word Segmentation and Named Entity Recognition}}
\author{\textbf{李大為 Da-Wei Lee} \\
Peking University \\
{\tt 1701210963@pku.edu.cn}}
\date{\vspace{-5ex}}

\begin{document}
\maketitle

\begin{multicols}{2}

\section{Introduction}
\label{sec:introduction}

\subsection{Word Segmentation}
\label{sec:cws}

In \fnurl{word segmentation}{https://en.wikipedia.org/wiki/Text_segmentation} task, we can reduce it to a 4-class classification problem.

The four classes:

\begin{itemize}
    \item B: Begin of a token
    \item M: Middle of a token
    \item E: End of a token
    \item S: single character as a token
\end{itemize}

Thus a seperated (labeled) dataset can be transform to the trainable descripiton - a word with a label, a sentence as a sequence.

For example: 小明~吃~冰淇淋 $\Rightarrow$ (小, B), (明, E), (吃, S), (冰, B), (淇, M), (淋, E).

\subsubsection*{Evaluation}
\label{sec:cws_eval}

The evaluation of the word segmentation task is not that simple, because the "word-level" measurement means nothing. So we need "entity-level" metrics.

There is an approach called the Golden Standard, the classic evaluation method is given by \fnurl{SIGHAN Bakeoff 2005}{http://sighan.cs.uchicago.edu/bakeoff2005/}.

We mark the words' start position and the end position then compare to the golden standard position. Then we can get the following counts:

\begin{itemize}
    \item $N$: Golden standard's word count
    \item $e$: Error (not in golden standard) word count
    \item $c$: Correct (in golden standard) word count
\end{itemize}

And use these numbers we can calculate precision ($P=\frac{c}{c+e}$), recall ($R=\frac{c}{N}$) and f1 score ($F_1=\frac{2\times P\times R}{P+R}$). We can also get "error rate" ($ER=\frac{e}{N}$) as one of the metrics.

\subsubsection*{Given Data}

Here is the summary of the given training data.

\begin{itemize}
    \item Sentences in training data: 95304
    \item Total unique word: 4743
    \item Max sentence (sequence) length: 165
    \item Classes to classify: 4
\end{itemize}


\subsection{Named Entity Recognition}
\label{sec:ner}

Named entity recognition task is very similar with word segmentation. We can also reduce it into a sequence labeling or a classification problem.

In our data we have three tags PER, LOC, ORG represent person, location, organization respectively. Each of them like word segmentation has begin and middle (but without end). Here is the labels:

\begin{itemize}
    \item B-PER: Begin of a person entity
    \item I-PER: In a person entity
    \item B-LOC: Begin of a location entity
    \item I-LOC: In a location entity
    \item B-ORG: Begin of a organization entity
    \item I-ORG: In a organization entity
    \item N: Not an entity (sometimes use O)
\end{itemize}

The given dataset is already transfered to the trainable format - a word with a label and sentences are sepreated with double new line character.

So in this task we can save the effort of transfer between trainable format and the final output.

\subsubsection*{Evaluation of Named Entity Recognition}
\label{sec:ner_eval}

The evaluation of named entity recognition is also a little tricky. And we need to also evaluate on entity-level.

Because, in the dataset, the data is almost labeled with N tag. If use word-level metrics we will get about 90\% precision even we predict N all the times.

There are plenty of evaluation metrics, such as \fnurl{CoNLL-2003}{https://www.aclweb.org/anthology/W03-0419} use only precision, recall and f1 score; \fnurl{SemEval-2013 Task 9-1}{https://www.cs.york.ac.uk/semeval-2013/accepted/76_Paper.pdf} use four advanced metrics (strict, exact, partial, type) to calculate precision, recall, f1 score.

And I'll use SemEval's metrics later on.



\section{Approach}
\label{sec:approach}

In this paragraph I'll introduce the approaches including tricks and models that I attempt to solve this task.

\subsection*{Data Preprocessing}

Because the sentence length of the training data are differ. But the model's input should be a fixed shape. Thus I have padded all the sentences into the maximum sentence length among both training data and test data. In the meanwhile, it is necessary to record the actual sequence length, when we will need to mask the rest of the part out when evaulation during training phase.

The padding is using an padding character, it is shared with Out-of-vocabulary character. The OOV happened when testing the test data that is possible appear words not in training set. Thus I have to replace it with OOV and refill it back when generating output file.

I tried to encode the words with one-hot encoding. (I was plan to try different embedding approach as input but the computation time is much longer than I expected thus finally decide to improve other part of models.) And in the first hand found that it is not possible to transfer entire dataset at once. By calculating the data size there will need more than 500GB memory which is not practical. Thus, I encode sentences only when needed (i.e. per batch).

\subsection{CRF}
\label{sec:crf}

CRF~\cite{lafferty2001conditional}

\subsection{BiLSTM with CRF}
\label{sec:bilstm_crf}

I found that the feature selection is very important to the CRF. One-hot encoding is not good at representing a word's meaning. The BiLSTM with CRF~\cite{huang2015bidirectional} approach is kind of a solution.

When using bidirectional RNN, we can consider it as a dynamic embedding layer that pack the meaning of a word. Use this as the input of the CRF will get much better result.

And because bidirectional, it can better extract the context among a word, compare with simple RNN or other naive embedding methods, that it can reduce the noise when input into the probabilistic graphical model.

The reason why we still need CRF instead of just using dense layer right after the BiLSTM layer. Lets consider the scenario in named entity recognition, all the I-[tag] should followed by B-[tag]. Using dense layer we cannot promise this rule. But with CRF, because there is no chance (zero probability) for a named entity with I-[tag] before a B-[tag], that is the transition probability on the transition matrix will be zero, thus it guarantees this rule establish.

In practice, I use dropouts to make the model more generalize that I use dropout wrapper on the RNN cell and set the dropout rate with 0.5 while training. And do not forget to disable the dropouts while doing inference.


\end{multicols}

\section{Experiment}
\label{sec:experiment}

As the requirement of the tasks. I split the labeled data into two part. 70\% for training set and 30\% for test set. The experiment result are the test on the test set. (And finally will use the entire labeled data to train another model and use it to label final submission).

\subsection{Word Segmentation Evaluation}
\label{sec:cws_eval}

The evaluation result check out Table~\ref{tab:cws_result}. We can found that the BiLSTM with CRF model is perform much better than pure CRF model.

\begin{table}[htbp!]
    \centering
    \resizebox{\textwidth}{!}{
        \begin{tabular}{lccccccc}
        \toprule
            \multicolumn{1}{c}{\multirow{2}{*}{Model}} & \multicolumn{4}{c}{Parameters}                              & \multicolumn{3}{c}{Test Set}     \\
            \multicolumn{1}{c}{}                       & Input Embedding & Train Epoch & Learing Rate & Dropout Rate & P         & R        & F1        \\
            \midrule
            CRF                                        & One-hot         & 100         & 0.01         & None         & 76.36     & 75.06    & 75.70     \\
            BiLSTM + CRF                               & One-hot         & 10          & 0.01         & 0.5          & 90.02     & 91.08    & 90.55     \\
        \bottomrule
        \end{tabular}
    }
\caption{Best evaluation result of word segmentation on the test set (in \%)}
\label{tab:cws_result}
\end{table}

\subsection{Named Entity Recognition Evaluation}
\label{sec:ner_eval}

The evaluation result check out Table~\ref{tab:ner_result}. The training parameter is basically the same except the learning rate. This part will be discussed in the next section.

\begin{table}[htbp!]
    \centering
    \resizebox{\textwidth}{!}{
        \begin{tabular}{lccccccc}
        \toprule
            \multicolumn{1}{c}{\multirow{2}{*}{Model}} & \multicolumn{4}{c}{Parameters}                              & \multicolumn{3}{c}{Test Set}     \\
            \multicolumn{1}{c}{}                       & Input Embedding & Train Epoch & Learing Rate & Dropout Rate & P         & R        & F1        \\
            \midrule
            CRF                                        & One-hot         & 100         & 0.001        & None         & TBD       & TBD      & TBD       \\
            BiLSTM + CRF                               & One-hot         & 10          & 0.001        & 0.5          & TBD       & TBD      & TBD       \\
        \bottomrule
        \end{tabular}
    }
\caption{Best evaluation result of named entity recognition on the test set (in \%)}
\label{tab:ner_result}
\end{table}

\section{Analysis and Discussion}
\label{sec:analysis}

Task analysis and discussion


\bibliographystyle{acm}
\bibliography{citedbibtex}

\end{document}